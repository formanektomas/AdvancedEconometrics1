\documentclass{beamer}
%
% Choose how your presentation looks.
%
\usepackage[T1]{fontenc}
\usepackage[utf8]{inputenc}
\usepackage{lmodern}  
\usepackage{amsmath}
\usepackage{bm}
%
% For more themes, color themes and font themes, see:
% http://deic.uab.es/~iblanes/beamer_gallery/index_by_theme.html
%
\mode<presentation>
{
  \usetheme{Darmstadt}      % or try Darmstadt, Madrid, Warsaw, ...
  \usecolortheme{default} % or try albatross, beaver, crane, ...
  \usefonttheme{serif}  % or try default, serif, structurebold, ...
  \setbeamertemplate{navigation symbols}{}
  \setbeamertemplate{caption}[numbered]
  \setbeamertemplate{headline}{}
}
%
%
\title[Week1]{Block 1 \\  Repetition from BSc courses \\ Time series regression models \\ Non-linear extensions to LRMs}
\author{Advanced Econometrics 4EK608}
\institute{Vysoká škola ekonomická v Praze}
\date{}

\begin{document}
 
\begin{frame}
  \titlepage
\end{frame}

% Uncomment these lines for an automatically generated outline.
\begin{frame}{Outline}
  \tableofcontents
\end{frame}

%------------------------------------------------------
\section{Estimation methods, predictions from a model}

\begin{frame}{Linear regression model (LRM) and OLS estimation}
$$
\bm{y} = \bm{X\beta} + \bm{\varepsilon}
$$
\textbf{LRM assumptions:}\\
(Notation follows Greene, Econometric analysis, $7^{\textnormal{th}}$ ed.)
\medskip
\begin{enumerate}
    \item[A1] \textbf{Linearity:} $y_i = \beta_1 + \beta_2 x_{i2} + \dots + \beta_K x_{iK} + \varepsilon_i$ \\LRM describes linear relationship between $y_i$ and $\bm{x}_i$.
    \item[A2] \textbf{Full rank:} Matrix $\bm{X}$ is an $n \! \times \! K$ matrix with rank $K$.\\ Columns of $\bm{X}$ are linearly independent and $n \geq K$.
    \item[A3] \textbf{Exogeneity of regressors:} $E[\varepsilon_i | \bm{X}]=0$ (strict form). \\If relaxed to contemporaneous form in TS: $E[\varepsilon_t | \bm{x}_t]=0$.\\Law of iterated expectations: $E[\varepsilon_i | \bm{X}]=0 ~\Rightarrow~ E[\varepsilon]=0$.\\Also, we assume disturbances convey no information on each other: $E[\varepsilon_i|\varepsilon_1,\dots,\varepsilon_{i-1},\varepsilon_{i+1},\dots,\varepsilon_n]=0$.
\end{enumerate}
\end{frame}
%------------------------------------------------------    
\begin{frame}{Linear regression model (LRM) and OLS estimation}
$$
\bm{y} = \bm{X\beta} + \bm{\varepsilon}
$$
\textbf{LRM assumptions (continued):}
\medskip
\begin{enumerate}
    \item[A4] \textbf{Homoskedastic \& nonautocorrelated disturbances:}$$E[\bm{\varepsilon\varepsilon}^{\prime}]=\sigma^2\bm{I}$$ Homoscedasticity: $\textnormal{var}[\varepsilon_i|\bm{X}]=\sigma^2, \qquad \forall~i=1,\dots,n$.\\Independent disturbances: $\textnormal{cov}[\varepsilon_t,\varepsilon_s|\bm{X}]=0, \qquad \forall~t \neq s$.\\ \smallskip GARCH models (i.e. $\textnormal{var}[\varepsilon_t|\varepsilon_{t-1}]=\sigma^2+\alpha \varepsilon_{t-1}$) do not violate the conditional variance assumption, but $\textnormal{var}[\varepsilon_t|\varepsilon_{t-1}] \neq \textnormal{var}[\varepsilon_t]$.
    \item[A5] \textbf{DGP of X:} Variables in $\bm{X}$ may be fixed or random.
    \item[A6] \textbf{Normal distribution of disturbances:} $$\varepsilon | \bm{X} \sim N[\bm{0}, \sigma^2\bm{I}].$$
\end{enumerate}


\end{frame}
%------------------------------------------------------
\section{Non-linear extensions to LRM, quantile regression}

\begin{frame}{Introduction}

\begin{itemize}
  \item 
\end{itemize}

\end{frame}
%------------------------------------------------------
\section{TS-based regression models, non/stationarity, cointegration}

\begin{frame}{Introduction}

\begin{itemize}
  \item 
\end{itemize}

\end{frame}
%------------------------------------------------------

\end{document}
